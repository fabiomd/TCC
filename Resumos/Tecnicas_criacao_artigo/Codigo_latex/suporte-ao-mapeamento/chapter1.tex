\chapter{Introdução}

\section{Motivação}

\paragraph{}
A sociedade criou normas e  que possibilitam a refutação de resultados. Permitiu o desenvolvimento do raciocínio crítico científico, ou  metodologia científica.

\paragraph{}
Um trabalho de pesquisa necessita de um estudo prévio de trabalhos existentes. 

\paragraph{}
Aspectos que podem dificultar a obtenção do conjunto mais correto possível de bibliográfia relacionadas: 

\subparagraph{}
Utilização de termos diferentes para denominar um mesmo assunto.

\subparagraph{}
Geração acelerada de produção científica.

\subparagraph{}
Ausência de um único concentrador de informações seobre novaa criações.

\subparagraph{}
O idioma no qual um certo documento foi escrito.

\section{Problema}

\paragraph{}
Atualmente  a bibliográfia proposta está completa?

\paragraph{}
Uma bibliográfia é completa se o autor considerou todo o universo relevate acerca do tema. 

\paragraph{}
Ao se passarem inúmero anos, essa bibliografia ainda estará "completa"?

\paragraph{}
Um estudo tende a ser tornar desatualizado, com o surgimento de novos trabalhos que agregam conhecimenteo que antes eram desconhecidos. Fazendo os estudos finalizados se tornarem base para um novo que irá corroborar, refutar ou superar os seus resultados.

\paragraph{}
Quão restrita a um tema deve ser a pesquisa?

\paragraph{}
Um estudo pode envolver diversas áreas do conhecimento. Podendo transitar por ciências diferentes.

\paragraph{}
Quão profundo deve-se pesquisar em um certo tema?

\paragraph{}
Há estudos que demandam conhecimento superficial sobre determinados temas, outros requerem esforço maior a fim de se obter conhecimentos mais específicos de um area.

\section{Proposta}

\paragraph{}
O método proposto busca alcançar o objetivo através:

\subparagraph{}
Relevância de cada referência envolvida.

\subparagraph{}
Agrupamento de referências por áreas semelhantes.