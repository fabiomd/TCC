\chapter{Revisão da Literatura}

\section{Revisão da literatura, tipos e conceitos}

\paragraph{}
É um procedimento onde no qual se adiciona de fato conhecimento à area pesquisada, utilizando dissertação de conceitos aprendidos, agregando conclisões, interpretações e questionamentos. Engloba a pesquisa bibliográfica.

\section{Revisões sistemáticas}

\paragraph{}
Define as etapas a serem seguidas, para realizar corretamente a pesquisa bibliográfica. Busca, através da pesquisa bibliográfica sistemática, cobrir os conceitos envolvidos no tema foco. Trata-se de uma busca para estabelecer o estado das evidências encontradas acerca do assunto em foco.

\section{Pesquisa bibliográfica}

\paragraph{}
Trata-se da realização da revisão literária. Antes de dissertar acerca de um assunto, é necessário obter o conhecimento sobre trabalhos existentes na area, termos e conceitos periféricos.

\section{Mapeamento sistemático}

\paragraph{}
Trata-se do método que visa identificar os estudos existentes acerca de um tema. Provem um estudo mais abrangente que a pesquisa bibliográfica.  Análisa o universo que circunda as relações entre as  publicações.

\section{Visualização de domínio do conhecimento}

\paragraph{}
Engloba múltiplos tipos de análise e representações de conhecimento. Dentre estão: cientométrica, bibliográfia e análises de citações.

\paragraph{}
Possui um estudo mais amplo que o mapeamento sistemático. Com objetivo de analisar as publicações envolvidads de forma qualitativa e quantitativa. Com foco em entender melhor áreas analisadas, observar tendências e compreender a dinâmica científica.