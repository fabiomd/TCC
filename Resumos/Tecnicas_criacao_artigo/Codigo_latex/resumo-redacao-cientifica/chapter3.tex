\chapter{Qualidade das Fontes da Pesquisa}

\section{Consulta bibliográfica}

\paragraph{}
A elaboracão de um trabalho científico, paper ou artigo exige apoio das próprias idéias com fontes reconhecidamente aceitas.
\paragraph{}
O Trabalho técnico ou cientifico, pressupõe vasta consulta bibliográfica, que somente se esgota após consulta exaustiva a obras de sustentação as ideias que pretende expor.
A biblioteca é um passo obrigatório para apresentação de textos bem fundamentados.
 
\section{Acervo}
 
\paragraph{}
Acervo é um conjunto de obras patrimônio ou documentos abrigados por uma biblioteca, são classificadas pelo sistema de Melvil Dewey ou pelo Sistema de Classificação Decimal Universal aplicam uma divisão por publicação, tipos de informação e use. Bibliotecas públicas dividem seus livros em coleção geral, livros de referência e períodicos

\section{Tipos de publicação}

\paragraph{}

Coleção Geral abarca livros científicos,didáticos e literários.
Referência englobam enciclopédias,dicionãrios, índices, atlas, bibliografias.
Periódicos sao publicações que apresentam artigos atuais.

\section{Tipos de informação}

\paragraph{}

As informações de uma publicação são primárias ou secudárias.
Primário consiste do texto original, podendo ser baseado em pesquisa ou criaçao do autor. O material primário compreende períodicos. O secundário compreende obras de referência.

\paragraph{}

Recomenda-se a busca por fontes primárias isentas de intrepretação.

\section{Quanto à utilização}

\paragraph{}

A divisão do acervo, baseia-se em obras de consulta e literárias corrente. Compreende obras literárias , divulgação científica e teórica.

\section{Uso da biblioteca}

\paragraph{}

A procura por obras se da por consulta ao catálogo do acervo bibliográfico.
Livros nas Estantes são organizados por assunto e autores.

\section{Biblioteca informatizada}

\paragraph{}

As bibliotecas estão equipadas com computadores que registram o acervo em arquivos eletrônicos.Assim o operador do terminal do computador indica elementos para a procura do documento. 






 
 
 