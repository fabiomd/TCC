\chapter{Resumo}
\section{Conceito do texto}
\paragraph{}
Texto é um tecido verbal estruturado de forma que idéias formem um todo coeso, uno, coerente. Todas as partes devem estar interligadas e manifetarem um direcionamento único.

\paragraph{}
Fragmentos que não constituem um texto. Falta de coerência entre afirmativa inicial e final.

\paragraph{}
Um texto pode ser constituído além de palavras, por desenho, charge e uma figura.

\paragraph{}
Um texto eficaz depende da competência do autor, da interação autor/leitor ou  emissor/receptor. Exige conhecimento do código, normas, gramáticas. Deve levar-se em conta o contexto.

\section{Contexto}

\paragraph{}
Contexto são informações que acompanham o texto, sua compreensão depende dessas informações. Deve ser visto em duas dimensões: estrutura de superfície e estrutura de profundidade. Estrutura de superfície considera elementos do enunciado. Estrutura profunda considera a semântica das relações sintáticas.

\paragraph{}
Produção e recepção estão condicionadas à situação. Contexto pode ser imediato ou situacional.

\paragraph{}
Contexto imediato são elementos que seguem ou precedem o texto imediatamente.

\paragraph{}
Contexto situacional são elementos exteriores ao texto. Acrescenta informações, quer históricas, geográficas, sociológicas, literárias, aumentando a eficácia da leitura.

\section{Intertexto}

\paragraph{}
Um texto pode ser produto de relações entre textos. Procedimentos intertextuais comuns são: paráfrase, paródia e estilização.

\paragraph{}
Paráfrase pode ser ideológica ou estrutural. Ideológica varia a sintaxe mantendo as idéias. Estrutural é recriação do texto e contexto. Comentário crítico, avaliativo, apreciativo, resumo, resenha, recensão são parafrásticas estruturais do texto.

\paragraph{}
Estilização exige recriação do texto, considerando procedimentos estílisticos.

\paragraph{}
Paródia o desvio é total, pode-se inverter idéias. Há uma ruptura, deformaçãp propositada, mostrando a inocência do texto, ou apresentando idéias omitidas pelo texto ou não expostas.

\section{Elementos estruturais do texto}
\paragraph{}
São elementos estruturais do texto: o saber partilhado, informação nova, provas, conclusão.

\paragraph{}
Saber partilhado é a informação antiga, do conhecimento da comunidade. Aparece na introdução para negoçiação com o leitor.

\paragraph{}
Informação nova é uma necessidade para existência do texto. O texto somente se configura quando veicula uma informação não conhecida pelo leitor, ou não era da forma que está sendo exposta. Informação nova não significa originalidade. Serve para desenvolver o texto, expandi-lo.

\paragraph{}
Saber compartilhado e informação nova não são suficientes para realização do texto. É presciso acrescentar provas, fundamentos das afirmações expostas.

\section{Resumo: a Norma NBR 6028:2003}

\paragraph{}
A Norma NBR 6028:2003, da Associação Brasileira de Normas Técnicas, define resumo como "apresentação concisa dos pontos relevantes de um documento". Apresentação sucinta, compacta, pontos importantes do texto.

\paragraph{}
Resumo é uma apresentação sintética e seletiva das idéias do texto, ressaltando a progressão e articulação. Abrevia o tempo dos pesquisadores; difunde informações para difundir e estimular a consulta do texto.

\paragraph{}
Partes constantes de um resumo: natureza da pesquisa, resultados e conclusões. Deve-se destacar o valor e a originalidade das descobertas.

\paragraph{}
A Norma da ABNT classifica resumos em: crítico, indicativo e informativo.

\paragraph{}
Resumo indicativo indica pontos principais do documento; não apresenta dados qualitativos e quantitativos. Conhecido como descritivo. Refere-se às partes mais importantes do texto.

\paragraph{}
Segundo a NBR 6028:2003, deve-se citar o uso de parágrafos no meio do resumo. Resumo é constituído de apenas um parágrafo.

\paragraph{}
Resumo informátivo despensa a leitura do texto original quanto às conclusões.

\paragraph{}
Resumo crítico denominado recensão ou resenha, redigido por especialistas compreendendo análise crítica do texto.

\section{Regras da apresentação}

\paragraph{}
Resumo ressalta o objetivo, método, resultados e conclusões. É precedido da referência.

\paragraph{}
O resumo é composto por uma sequência de frases concisas e afirmativas.

\paragraph{}
A primeira frase deve explicar o assunto do texto. Em seguida, especificar a categoria do tratamento: memória, estudo de caso, análise da situação ou ensaio.

\paragraph{}
Frases são compostas com verbo em voz ativa e terceira pessoa do singular.

\section{Técnicas de elaboração do resumo}

\paragraph{}
Rebeca Peixoto da Silva e outros indicam que para resumir é fundamental compreender sua organização. Apreende-se o todo pela leitura global do texto, com objetivo de compreender o texto em seu conjunto. A elaboração do resumo exige habilidade de leitura e escrita. O resumo permite compreender as idéias expostas. Resumir é um processo que compreende: encotrar a idéia-tópico do parágrafo. Se a idéia estiver subentendida será necessário isolar as frases-chaves para encontrar a idéia central. Em seguida, elimina-se as idéias secundárias ou não essenciais para a compreensão da idéia central.

\paragraph{}
Não cabe no resumo comentários ou julgamentos apreciativos. A dificuldade de resumir pode advir da complexidade ou competência do leitor.

\paragraph{}
A referência do texto é a documentação da pesquisa bibliográfica. A documentação pode ocorrer através de transcrições, resumos, síntese e referências. 

\paragraph{}
A situação inicial consiste-se da documentação: transições, resumos, síntese e referências. 

\paragraph{}
A informação nova é o estabelecimento do uso de documentação: uso de transcrições, resumo, síntese, simples referências. Transcrição textual justica-se quando há a necessidade de uma prova. Resumo tem função instrumental é usada quando não há propria biblioteca a obra utilizada. Síntese consiste na exposição das idéias centrais do texto. Referências são utilizadas em obras conhecidas ou de acesso facil.

\paragraph{}
As justificativas para realização de pesquisa documental resume-se em: necessidade de provas; obras de bibliotecas públicas; realização do objetivo da pesquisa.

\paragraph{}
Conclusão ressalta o trabalho científico.