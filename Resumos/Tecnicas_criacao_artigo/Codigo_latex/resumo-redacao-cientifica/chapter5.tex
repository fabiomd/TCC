\chapter{Estratégias de Leitura}
\section{Leitura e suas técnicas}
\paragraph{}
Segundo a pesquisadora Dolores-Durking pouca atenção é dada pelos professores para desenvolver a compreenção de textos escritos. Talvez se deva ao fato que enganosamente, país, professores e alunos considerarem o ensino da leitura restrita a alfabetização e à escola fundamental.
\paragraph{}
Um leitor competente reconhece a incompletude do discurso, considerando pressuspostos e subentendidos, o contexto situacional e histórico, a intertextualidade explicita a formação discursiva e ideológica.
\paragraph{}
Segundo Molina um bom leitor é capaz de praticar os niveis de leitura propostos por Mortíner J. adler e Charles Van Doren em como ler um livro: leitura elementar, leitura inspecional, leitura analítica, leitura sintópica. Um leitor competente transita pelos quatro níveis, com desenvoltura e autonomia.
\paragraph{}
Não há leitura tão-somente, mas diferentes leituras. Há interação entre leitor e texto possibilitando a identificação de multiplos significados. Textos diferentes exigem diferentes estratégias de leitura.
\paragraph{}
Uma prática de leitura é a tecnica SQ3R, de Morgan e Deese, compreende cinco etapas:
\subparagraph{}
Survey
\subparagraph{}
Question
\subparagraph{}
Read
\subparagraph{}
Recite
\subparagraph{}
Review
\paragraph{}
Molina apoiando-se nessas técnicas, propõe que a leitura siga os demais passos:
\subparagraph{}
Visão geral do capítulo
\subparagraph{}
Questionamento despertado pelo texto
\subparagraph{}
Estudo do vocabulário
\subparagraph{}
Linguagem não Verbal
\subparagraph{}
Essência do texto
\subparagraph{}
Síntese do texto
\subparagraph{}
Avaliação

\section{Visão geral do capítulo}

\paragraph{}
O leitor verificará a estrutura do capítulo, títulos e subtítulos. Observando grifos, itálicos, tamanho e estilo dos caracteres.

\section{Questionamento despertado pelo texto}

\paragraph{}
Faz-se  um levantamento de perguntas, sem buscar responde-las. Ensina Molina, "Para engajar-se numa leitura ativa é muito importante que o estudante saiba fazer perguntas, a fim de fortalecer a expectativa que forma em relação ao que vai encontrar no capítulo" acrescenta " Questionar é um hábito, e como tal deve ser cultivado".

\section{Estudo do vocabulário}
Para Molina, uma forma de despertar o prazer a leitura e consolidar o hábito a ler é oferecer textos interessantes, difíceis levando o leitor a aceitar o desáfio neles implícitos. Para ampliar o vocabulário o uso do dicionário, o emprego de palavras novas e sua análise. Uma forma de conhecer o significado das palavras é através do contexto. O significado também pode ser deduzido atentando-se ao contraste de idéias salientadas. Caso o contexto e a análise não expliquem o significado corre-se ao dicionário, devendo ler o verbete.

\paragraph{}
O estudante consultar o dicionário, examinando primeiramente as páginas introdutórias, a abreviaturas e outras. Considerando que as palavras estão em ordem alfabéticas e que as páginas possuem um cabeço. Cabe dinstinguir termo de palavra.

\paragraph{}
Ao estudar um texto, é preciso atenção aos termos empregados, desenvolvendo o vocabulário técnico. Termos técnicos são gravados em itálico, negrito, em caracteres maiúsculos, ou entre aspas, ou outro destaque.

\paragraph{}
O estudo do vocabulário não restringe a estudantes de letras. Livros de ciências humanas ou exatas, artes ou religião, pode oferecer dificuldades vocubulares

\paragraph{}
O leitor buscará a palavra do dicionário, lendo o verbete e anotando a palavra de sentido mais aproximada para o texto. Finalmente limitando-se a procurar uma palavra tão somente, verificar palavras da mesma família etimológica.

\paragraph{}
A listagem de palavras não é completa, visa aprender palavras e conjunto isoladamente, outro procedimento adequado é pesquisar a etimologia da palavra. Aprendida a nova palavra, é preciso empregá-la.

\paragraph{}
Esse estudo pode ser complementado por pesquisa sobre formação de palavras, constante das gramáticas de Língua Portuguesa. Rever a história da formação das palavra é um processo interessante para realizar o estudo.

\section{Linguagem não verbal}

\paragraph{}

O texto pode apresentar ilustrações. Não deve-se passar superficialmente, é preciso observá-las para entendê-las.

\section{Essência do texto}

\paragraph{}

A busca do conteúdo profundo apenas se concretiza após os passos anteriores; estudo da linguagem não verbal, questionamento do texto, visão geral do capítulo.

\paragraph{}
Exigências desse estágio da leitura:

\subparagraph{}
apreender as proposições do autor
\subparagraph{}
conhecer os argumentos do autor
\subparagraph{}
identificar a tese do autor
\subparagraph{}
avaliar as idéias expostas

\paragraph{}
O estudante aplica-se na copreensão das idéias dos parágrafos. Para Molina, "quando o leitor é capaz de encontrar o tópico frasal de cada parágrafo, já tem bastante adiantada a tarefa de resumir, por sua própria conta, o texto lido".

\paragraph{}
Nessa etapa o leitor deve sublinhar o texto, com parcimônia, com economia antecipando a sublinha.

\paragraph{}
A avaliação do texto compreende: validade das idéias, completude, correção de argumentos, coerência e suficiência das provas, consecução dos objetivos prometidos.

\section{Resumo do texto}

\paragraph{}

Resumo é a recriação do texto original realizado após análise do texto, divisão por partes principais e distinção do essência e não essência. Exige compreenção profunda do texto.

\paragraph{}
As etapas levam ao aprimoramento do estudo, cujo objetivo e incorporar conhecimentos e avaliá-los.

\paragraph{}
A síntese oral e escrita produz melhores resultados. Molina: " a exposição oral deve ser a oportunidade para que ele [leitor] coloque em ordem suas idéias e teste esta ordenação ao passá-la para seus colegas". 

\paragraph{}

Resumo de textos descritivos pede pensamento visual e espacial; narrativos demanda atenção quanto aspectos casuais e sequenciais; Dissertativo por pensamentos lógicos-abstratos. Na dissertação é importante a organização e hierarquização das idéias. A atenção deve ser concentrada no uso de expressões comparativas, contrastes, concessivas, sequência ou cronologia dos acontecimentos. causa e consequência dos fatos.

\section{Avaliação}

\paragraph{}
A preocupação é salientar a necessidade de orientação quanto a capacidade crítica. Levando à independência de pensamento crítico. 

\paragraph{}
A avaliação engloba respostas do leitor e respostas oferecidas pelo texto.

\paragraph{}
A crítica é subsequente ao entendimento do texto. Molina: "Se o leitor entendeu realmente o livro, nada impede que ele concorde ou discorde do autor". Adler e Van Doren : "Concordar sem entender é inépcia. Discordar sem enteder é impertinência".

\paragraph{}
O estudo do texto completa-se ao descobrir as idéias do autor e sua tese defendida.

\section{Tipos de leitura}

\paragraph{}
A leitura classifica-se em : Skimming e scanning. Skimming capta a tendência da obra. O leitor vale-se da leitura superficial de títulos, subtítulos e parágrafos. A leitura do significado busca a visão geral do texto. A leitura engloba ler, reler, anotar e resumir. Leitura crítica envolve reflexão, avaliação e comparação. Leitura classificada caracteriza-se por procurar tópicos da obra.

\paragraph{}
Classificações sao muitas  e variadads; envolvendo aspectos formais; aspectos de conteúdo. Leitura com objetivo agaria informação, dados e fundamentações para base do trabalho cientifíco. Indica-se a leitura informativa podendo subdividir-se em reconhecimento, seletiva, crítica e interpretativa.

\paragraph{}
Leitura de reconhecimento proporciona uma visão geral da obra; permitindo encontrar informações de que necessita. A leitura seletiva seleciona as informações necessárias.A Leitura crítica exige preocupações; exige esforço reflexivo. A leitura interpretativa relaciona informações do autor com problemas ao quais se busca um resposta.

\section{Aproveitamento da leitura}

\paragraph{}
O sentido do texto não é produzido apenas pelo autor. O leitor também produz sentidos. Compreender não significa atribuir ou descobrir o sentido passado pelo autor, significa reconhecer os mecanismos de funcionamento do discurso, do processo, significação chegando a uma leitura polissêmica.

\paragraph{}
A leitura pode ser facilitada por estratégias realizadas no processo. O diálogo entre autor e leitor surgirá entre outros contexto gerando textos que se relacionam. Como o pesquisador reproduz as informações colhidas no contexto sociocultural, segundo determinações historicas, atento ao precesso de significado, de constituição do discurso, consciêncizando que a ciência é resultado de formaçoes ideológicas e discursivas.

\paragraph{}
A esquematização do texto facilita a aprendizagem e retenção de informações.

\paragraph{}
A profussão de obras impõe ao pesquisado uma seleção. É imperativo do objetivo em vista. A seleção preocupa-se com obras a serem lidas, autores preferenciais, edições críticas e edições recentes.
A ultima edição revista pelo autor é preferida.
\paragraph{}
Em traduções a escolha será por obras fiéis ao texto do autor.

\section{Eficiência e eficácia na leitura}

\paragraph{}
Pessoas tem dificuldade de apreensão do que lêem, deve-se a velocidade de leitura. Retornado ao parágrao ou idéia precedene, prejudicando a compreensão e disperdiçando tempo.

\section{Ambiente}

\paragraph{}
A eficiência e eficácia da leitura depende do ambiente. Fatores como iluminação, arejamento, ventilação, ausênsia de ruídos.

\section{Objetivo da leitura}

\paragraph{}
A leitura busca assimilar o conhecimento e preparação itelectual. Para a concretização dos objetivos é necessária a busca da idéia mestra, o tópico frasal.

\paragraph{}
É necessário exercícios que pratiquem a identificação da idéia principal e hierarquização das secundárias. Melhorando a qualidade de leitura, com objetivo de captar, reter, integras conhecimentos para reformulá-los, recriá-los e transformá-los. 
Também indica-se a paráfrase.

\section{Compreensão do texto}

\paragraph{}
Enilde L. de J. Faulstich afirma a existência de textos inteligíveis e cujo conteúdo não é compreensível pelo leitor.

\paragraph{}
Dado o fato, a autora divide a leitura em informativa e interpretativa.

\paragraph{}
A leitura informativa compreende: seleção de idéias-chaves e critíca. A expressão-chave do paragráfo é compreendida pelo tópico frasal. Segundo Faulstich quando identificada a palavra-chave, busca-se as palavras-chaves secundárias.
\paragraph{}
A seleção de palavras-chaves deve ocorrer em todos os parágrafos. Pois possibilitam a elaboração do resumo do texto.

\paragraph{}
A leitura crítica exige reconhecimento da pertinência dos conteúdos apresentados.

\section{Segmentação textual}

\paragraph{}
A segmentação pode ser feita por quatro possibilidades: por espaço, tempo, personagens ou temas. Visando clarificar as relações entre as partes de um texto e amplificar a eficácia da elitura.

\subsection{Segmentação por tempo}

\paragraph{}

A divisão do texto por cronologia dos acontecimentos permite ao leitor o domínio sobre as transformações ocorridas. Em narrativas é relevante a transformação dos personagens, residindo seus significados.

\subsection{Segmentação por espaço}

\paragraph{}
A segmentação por espaço destaca diferenças ou semelhanças de acontecimentos. Permitindo comparar ocorridos.

\subsection{Segmentação por personagens}

\paragraph{}
As personagens e suas ações são essenciais na narrativa. Sem elas o leitor ficará privado de seu signifado.

\subsection{Segmentação por temas}

\paragraph{}
A segmentação por temas distingue idéias com intuíto de hierarquização e percepção como estrutura.

\paragraph{}
Inicialmente distingue-se as idéias principais das secundárias; em segui distingue-se as secundárias entre si. Analisando coexão das idéias e ordenação.

\section{Leitura interpretativa}

\paragraph{}
Segundo Faulstich a leitura interpretativa exige domínio da leitura informativa. É necessário a capacidade de compreensão, análise, síntese ,avaliação e aplicação.

\paragraph{}
Compreensão é a capacidade de entendimento literal da mensagem. O leitor ve o texto segundo a óptica do autor, identificando a tese do autor a ser defendida e o objetivo do texto.

\paragraph{}
Análise envolve a capacidade de verificação de partes constitutivas do texto, percebendo nexos lógicos das idéias e organização. 

\paragraph{}
A síntese implica capacidade de aprendazem das idéias essenciais do texto. Buscando reconstruir o texto, eliminando o secundário.

\paragraph{}
Avaliação entende a capacidade de emissão do juízo valorativo.

\paragraph{}
Aplicação caracteriza-se como capacidade para, com base no texto. Resolvendo situações semelhantes. Possibilitando a projeção de novas idéias e obtenção de resultados.

\section{Leitura crítica}

\paragraph{}
Exige conhecimento do assunto. Faz-se um levantamento dos tópicos frasais de todos os parágrafos. Buscando estabeler falhas ou fundamentos na hierarquização das idéias.

\section{Análise do texto}

\paragraph{}
Analisar significa decompor, examinar os elementos que compõem o texto. Com objetivo de penetrar nas idéias do autor e compreender a organização. É importante indicar os tipos de relações existentes entre as idéias expostas.

\paragraph{}
Desenvolve-se atravéz da explicação, discussão e avaliação. Com objetivo aprender a ler, escolher textos significativos, reconhecer a organização do texto, interpretá-lo, procurar o significado de suas palavras, desenvolver a capacidade de distinguir fatos, opiniões, hipóteses, detectar ideias principais e secundárias, chegando a uma conclusão.

\paragraph{}
O procedimento inicia-se pelo escolha do texto, apos deve-se lê-lo, relê-lo, esclarendo palavras desconhecidas. Nova leitura é feita para compreensão como um todo. Detecta-se a idéias do texto localizando outras idéias, comparando-as.

\section{Tipos de análise}

\paragraph{}
Os tipos de análise são: dos elementos, relações e estrutura do texto.

\paragraph{}
Análise dos elementos são as referências bibliográficas, estrutura do plano do livro ou texto, vocabulário, modelo teórico, doutrinas, idéias principais e secundárias, juízos expostos, conclusões.

\paragraph{}
Análise das relações é a busca de relações entre hipótese, provas e conclusões. Verificando a coerência dos elementos das partes do texto. Um texto oferece relações entre idéias principais e secundárias, confirmadas pelas opiniões exaradas, causas e consequências.

\paragraph{}
Análise da estrutura é o estudo das partes, buscando relações com o todo. Percebendo a inteção do autor, posição diante dos fatos. Preocupa-se com a posição do autor, conceitos adotados, estabelecimentos ilações, método de trabalho exposto.

\paragraph{}
Existe análise textual, temática, interpretátiva, deproblematização, síntese. Textual busca o levantar os elementos importantes, análise temática apreende o conteúdo, problemas alinhados, idéias expostas, qualidade da argumentação. Análise interpretátiva explica a posição do autor, detectando influências, e faz uma exposição crítica e avalia o conteúdo da obra. A análise de problematização levanta os problemas do texto, discutindo-os, e a síntese elabora um novo texto, após reunir os elementos e refleti-los.

\paragraph{}
É possivel estabelecer um roteiro de análise compreendendo: verificação das fontes, bibliográfia, metodologia utilizada, deficuldade relatadas pelo autor, reflexão sobre o texto,abrangendo análise e interpretação da obra. Deve constar do roteiro de leitura as sugestões proporcionadas pelo texto.

\section{Leitura na prática de redação}

\paragraph{}
A leitura aprimora a redação. Há dois tipos de leitura: informativa e interpretativa. Informativa compreende seleção e crítica.

\paragraph{}
Seleção refere-se a identificação da palavra-chave dos paragráfo. Em torno da palavra-chave o autor desenvolve sua idéia principal, que se identifica com o tópico frasal.

\paragraph{}
Porém não se estrutura apenas com idéia-chave, tópico frasal, idéia principal; também apresenta idéias secundárias.

\paragraph{}
Após reconhecida a palavra-chave do tópico frasal, identifica-se as palavras-chave secundárias que constituem o tópico frasal.
