\chapter{Paráfrase e Citações Diretas e Indiretas (NBR 10520:2002)}

\section{Conceito de paráfrase}

\paragraph{}
Para Greimas e Courtés paráfrase "consiste em produzir, no interior de um mesmo discurso, uma unidade discursiva que seja semanticamente equivalente a uma outra unidade produzida anteriormente". O sentido é equivalente ao do texto original. 

\paragraph{}
Parafrasear traduz as palavras de um texto com sentido equivalente, mantendo as idéias originais.

\section{Tipos de paráfrase}

\paragraph{}
Reprodução, comentário explicativo, resumo, desenvolvimento e paródia são formas parafrásticas.

\paragraph{}
Reprodução implica reescrever o texto, substituindo os vocábulos.

\paragraph{}
Comentário explicativo ou explanação de idéias, desenvolve conceitos, ampliando idéias.

\paragraph{}
Desenvolvimento ou amplificação de idéias, consiste em reescrevê-lo complementando com exemplos, pormenores, comparações, contrastes, exposições de causa e efeitos e definições dos termos utilizados.

\paragraph{}
Resumo formado por excelência de paráfrase.

\paragraph{}
Paródia é uma composição literária imitando o tema ou forma da obra, explorando aspectos cômicos, quer expondo aspectos satíricos.

\section{Citações}

\paragraph{}
Citação é a "menção, no texto, de uma informação extráida de outra fonte".

\paragraph{}
Podem ser diretas ou indiretas. As primeiras são transportadas como se apresentam na fonte; as segundas mantêm o conteúdo do texto original mas são parafraseadas.