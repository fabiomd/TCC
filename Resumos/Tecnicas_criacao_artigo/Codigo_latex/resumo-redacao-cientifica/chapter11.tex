\chapter{Publicações Científicas}

\section{Introdução}

\paragraph{}
São publicações cientificas: artigo científico, comunicação cientifica, ensaio, informe científico, paper, resenha críticica e dissertações científicas.

\section{Artigo científico}

\paragraph{}
Trata de problemas científicos. Apresenta o resultado de estudos e pesquisas. Permite que as experiências sejam repetidas. Compostas por: título do trabalho, autor, credenciais do autor, lcal das atividades; sinopse; corpo do artigo; parte referencial.

\section{Comunicação científica}

\paragraph{}
São informações apresentadas em congressos, simpósios, reuniões, academias, sociedade científicas.

\section{Ensaio}

\paragraph{}
É uma exposição metódica dos estudos realizados e de conclusões originais apuradas após exame de um assunto.

\section{Informe científico}

\paragraph{}
É um relato escrito que divulga resultados parciais ou totais de uma pesquisa.

\section{Trabalho científicos}

\paragraph{}
São textos elaborados segundo estrutura e normas preestabelecidas, podendo ser resenhas, informe científico, artigo científico, monografia ou paper.

\section{Monografia}

\paragraph{}

É uma dissertação que trata de um assunto particular, de formar sistemática e completa.

\section{Dissertação}

\paragraph{}
É um texto realiado segundo o molde da teste. Tem finalidade apenas didática.

\section{Tese}

\paragraph{}

É o relato de pesquisa exigido para obtenção do grau de doutor. Sua estrutura é de monografia.

\section{Paper}

\paragraph{}

É uma síntese de pensamentos aplicados a um tema específico. Deve ser original e reconhecer a fonte do material utilizado.

\section{Trabalho de conclusão de curso (TCC)}

\paragraph{}

É um trabalho científico dissertativo e monográfico.

\section{Projeto de pesquisa}

\paragraph{}

Procede o relatório de pesquisa, inclui os elementos estruturais do texto monográfico.