\chapter{Prática da Leitura}
\section{Conceito}

\paragraph{}

A linguagem não pode ser estudada separadamente da sociedada que a produz pois é constituida de processos históricos.
A leitura é produzida quando o leitor interage com o autor do texto. Deve levar em conta aspectos da linguagem.
\paragraph{}
A legitibilidade do texto depende da coesão, da coerência, da sinalização de tópicos e da relação entre leitor e autor do texto.

\section{Leitor e produção da leitura}

\paragraph{}

A leitura é seletiva deve-se realiza-la por: 
\subparagraph{}
relevância do texto
\subparagraph{}
relação do texto
\subparagraph{}
relevancia do autor e seu referente
\subparagraph{}
relação texto leitor

\paragraph{}

O texto é uma unicidade e o contexto é a situação do discurso ou conjunto de circunstâncias de um ato de enunciação.
Envolvendo o ambiente físico e social, entendem-se por situações ou acontecimentos que precedem o ato de enunciação.

\paragraph{}
A compreenção do texto deve-se ao processo de interação. Na ideologia há um intercolutor no ato da escrita "virtual" e um "real" havendo entre eles um debate de idéias. A leitura constitui-se de um momento da constituição do texto. Os interlocutores desencadeiam o processo de significação do texto.

\section{Fatores que constituem as condicões de produção da leitura}

\paragraph{}

O sentido do texto provém de sua formação discursiva, rementendo-se a uma formatação ideológica. A formação ideológica constitui-se por um conjunto de atitudes e representações não individuais que reportam as posições de classe. A formação discursiva relaciona-se com a formação ideológica. Dependendo da formação ideologica da classe pertencente as palavras ganham ou mudam os sentidos, sendo assim necessária a verificação das concepções correntes na época e sociedade produzida.

\paragraph{}

As condições de produção da leitura são a relação dos textos, situação e interlocutores. Enquanto explicita-se o funcionamento do discurso, mostra-se o funcionamento de um texto.

\paragraph{}

O funcionamento do discurso exige levantamento das condições de produção remetendo-se a exterioridade do discurso. A incompletude de qualquer discurso deve-se a multiplicidades de sentidos possíveis.Logo o texto não resulta-se da soma de segmentos, frases ou resultados da soma de interlocutores. O sentido do texto resulta da situação discursiva. Deve-se preencher espaços do texto gerada por implícitos. Intertexualidade é a relação entre textos.
\paragraph{}
Pressupostos são idéias não expressas explicitamente percebidas atrávez do contexto da leitura, suas afirmações não podem ser discutidas pelo leitor.
\paragraph{}
Subentendidos são situações contidas por trás de uma afirmação, ao contrário do pressuposto este depende do ouvinte e leitor.
\paragraph{}
A estratégia de leitura depende do tipo do discurso.
Discurso lúdico tende a polissemia, o autoritário a paráfrase e o polêmico o equilíbrio.
\paragraph{}
O leitor deve reconhecer os tipos de discurso e estabelecer a relevância à sua leitura. 
Ao contexto deve levar-se em consideração o sujeito enunciado, o anunciação e o textual. Durante o exame do sujeito oracional, o leito deve identificar o sujeito, explícito ou oculto. Quanto ao sujeito de enunciação deve-se identificar sua perspectiva e sua perpeção textual. 
É preciso considerar os tipos de leitores. 

\section{A aliança liberal e a revolução de 1930}

\paragraph{}
A sucessão do Presidente Washington Luís causou uma ruptura no sistema dominante como acontecera em 1910 com a Campanha Civilista de Ruy Barbosa e em 1922 com a Reação Republicana.

\paragraph{}
Seguindo os ditames da política "café com leite", Washington Luís deveria apresentar a candidatura do Presidente de Minas Gerais. Antônio Carlos de Andrada, à presidência da República. Não simpatizando com o polìtico. Washington Luís fez Júlio Prestes de Albuquerque, Presidente de São Paulo, em 1928 e após candidato à Presidente do Rio Grande do Sul e lançou a candidatura de Getúlio Vargas, com companheiro de chapa João Pessoa, Presidente da Paraíba. Com apoio de três Estados - Minas Gerais, Rio Grande do Sul e Paraíba, formando a aliança Liberal. Grandes oradores percorreram o país defendendo a oposição.

\paragraph{}
Procedidas as eleições, com fraudes em ambos os lados, foi eleito Júlio Prestes. Os grupos radicais não conformados com os resultados, liderados por Juarez Távora, João Alberto, Oswaldo Aranha e Góis Monteiro prepararam uma Revolução. Eclodindo nos dias 3 e 4 de outubro nos estados liberais depois expandindo-se para os demais, forçando a fuga ou abandono de poder dos governantes. Em 24 de outubro Washington Luís foi deposto por uma Junta Militar no Rio de Janeiro, Getúlio Vargas assumiu o Governo em 3 de novembro. As situações estatuduais desmoronaram devido a falta de apoio popular aos perrepistas, partidários de Júlio Prestes.
\paragraph{}
A plataforma de Vargas, lida na Esplanada do Castelo no Rio de Janeiro durante a campanha eleitoral não era das mais avançadas. Na época em que a Europa enfrentava uma luta entre comunistas e socialistas.
\paragraph{}
Vargas assumiu o governo, jogando entre correntes opostas, de um lado estavam jovens oficiais, chamados de "tenentes" que formavam a corrente do "tenentismo" e de outro lado estava o grupo de velhos políticos, enquadrados anteriormente na República Velha. Procurou Vargas aquinhoar as duas correntes.
\paragraph{}
Após os primeiros momentos, os partidos políticos que apoiavam o Libertador no Rio Grande do Sul, o Democrático em São Paulo e as dissidente do Partido Republicano, prescionaram o Governo para que seja convocada uma Assembléia Constituinte para restaurar o Estado de Direito, enquanto os líderes radícais e os tenentistas uma série de reforams e punindo os políticos apeados do poder.



 
