\chapter{Fichamento}
\section{Regras do jogo}
\paragraph{}
Ao estudioso pede-se um levantamento bibliográfico. A armezenação da informação deve ser realizado em arquivos de fichas ou pelo computador. Outros arquivos importantes são: arquivo de leitura, idéias e citações.

\paragraph{}
Arquivo de leitura é o registro de resumos, opiniões, citações. Arquivo bibliográfico registra livros a ser localizados, lidos, e examinados. Referências devem seguir as normas da ABNT, NBR, 6023:2000.

\paragraph{}
As fichas são recursos de estudo que se valem os pesquisadores para realização de uma obra didática, ciêntifica e outras.

\paragraph{}
As anotações que ocupam mais de uma ficha têm o cabeçalho reescrito. Fichas compreendem cabeçalho, referências bibliográficas, corpo da ficha e local da obra. Cabeçalho engloba título genérico ou específico e letra indicativa da sequência das fichas se for utilizada mais de um vez.

\paragraph{}
O fichamento é procedido por uma leitura atenta. Afastando-se da categoria do emocional e alcançando o nível da racionalidade, compreende; capacidade de analisar o texto, separa partes, examinar inter-relações, e resumo de idéias.

\section{Fichas de leitura}

\paragraph{}
São fichas que registram informações bibliográficas completas, anotações sobre tópicos, citações diretas, juízos valorativos, resumo, comentários. A indicação de referências bibliográficas segue normas da ABNT.

\section{Fichamento de transcrição}

\paragraph{}
Citações de até três linhas devem ser contidas entre aspas duplas. Aspas simples indicam citações no interior da citação.

\paragraph{}
Deve indicar o numero da página onde transqueve-se. Erros gramaticais devem ser mantidos, apenas adicionado (sic) para idêntifica-los.

\paragraph{}
Supressão de palavras são indicados por três pontos entre colchetes. ex: [...]. Supressões iniciais e finais não prescisam ser indicadas.

\paragraph{}
Citações com mais de três linhas devem ser destacadas com recuo de 4 cm da margem esquerda, com letra menor que a do texto utilizado e sem aspas.

\section{Fichamento de resumo}

\paragraph{}
Resumo é uma redação informativo-referencial que reduz um texto a suas idéias principais. Resumo é uma parafrase não deve conter comentários e engloba duas fases: compreensão do texto e elaboração de um novo. Compreensão implica análise do texto e checagem de informações colhidas.

\paragraph{}
Compreensão das idéias deriva de dois métodos: análítico e comparativo.

\paragraph{}
Método analítico demanda atenção com os instrumentos linguísticos de coesão e com marcadores de tópicos discrusivos.

\paragraph{}
Método comparativo demanda atenção com a estrutura geral do texto e com as informações que respondem às expectativas criadas pelo leitor.

\paragraph{}
Informações da memória são orientadoras, guias para compreensão.

\paragraph{}
Supressão elimina palavras secundárias do texto. Atém-se a advérvios, adjetivos, preposições e conjunções.

\paragraph{}
Generalização substitui elementos específicos por genéricos.

\paragraph{}
Seleção elimina obviedades ou informações secundárias atendo-se às idéias principais.

\paragraph{}
Construção é uma paráfrase, respeitando as idéias do texto original.

\paragraph{}
Caso uma frase necessiar de explicações complementares, o autor pode utilizar suas palavras entre colchetes.

\section{Fichamento do comentário}

\paragraph{}
Segundo Francisco Gomes de Matos devem ser analisadas os aspectos quantitativos e qualitativos. Respondendo do texto sua constituição, conceitos abordados. Aspectos qualitativos recomenda-se ater-se a análise e detecção da hipótese do autor, objetivo, motivos de ter escrito o texto, idéias que o fundamentam.

\section{Fichamento informatizado}

\paragraph{}
A difusão de microcomputadores e processadores facilitou-se o armezenamento de informações em arquivos eletrônicos, não há limite de linhas, é possivel copias textos, transferir informações, localizar expressões-chaves.