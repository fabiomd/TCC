\chapter{Resenha}

\section{Que é resenha?}

\paragraph{}
Para Andrade, resenha é um trabalho que demanda cochecimento do assunto, para estabelecer comparações entre obras de mesma área e maturidade intelectual com intuíto de avaliação ou emitir juízo de valor.

\paragraph{}
Resenha é um relato minucioso das propriedades de um objeto, ou partes constitutivas; é uma redação técnica incluindo varias modalidades de texto: descrição, narração e dissertação. Estruturalmente descreve as propriedades da obra, relata credenciais, resume a obra, apresenta suas conclusões e metodologia emprega, expondo um quadro de referência.

\paragraph{}
Resenha crítica inclui textos com objetivo conduzir o leitor para informações puras. Combina resumo e julgamento de valor. A NBR 6028:2003, denomina resenha como resumo crítico. Com objetivo oferecer informações para que o leitor decida à consulta ou não do original. Deve resumir as idéias da obra, avaliar as informações contidas e a forma como são expostas e justificaar a avaliação realizada.

\section{Resenha descritiva e crítica}

\subsection{Resenha descritiva}

\paragraph{}
Para Fiorin e Platão , "resenhar significa fazer uma relação das propriedades de um objeto, enumerar cuidadosamente seus aspectos relevantes, descrever as circunstâncias que o envolvem".

\paragraph{}
Estrutura da resenha descritiva

\subparagraph{}
nome do autor
\subparagraph{}
título e subtítulo da obra
\subparagraph{}
se tradução, nome do tradutor
\subparagraph{}
nome da editora
\subparagraph{}
lugar e data da publicação da obra
\subparagraph{}
número de páginas e volumes
\subparagraph{}
descrição sumária de partes, capítulos, índices
\subparagraph{}
resumo da obra,salientando objeto, objetivo, gênero
\subparagraph{}
tom do texto
\subparagraph{}
métodos utilizados
\subparagraph{}
ponto de vista que defende

\section{Comentários sobre os elementos estruturais da resenha}

\paragraph{}
O título da obra é sublinhado. Utiliza-se inicial maiúscula apenas no primeiro nome do título, os demais nomes são grafados com letra minúsculas. Subtítulo é separado do título por dois pontos. O subtítulo não é sublinhado. Aparece na capa do livo em caracters menores que o título principal.

\subparagraph{}
O local de publicação é importante.

\subparagraph{}
A editora da obra é irelevante. 

\subparagraph{}
O ano de publicação interessa ao pesquisador.

\subparagraph{}
Informações sobre edição são importantes.

\subparagraph{}
O número de páginas é informação interessante com ela o pesquisador pode criar expectativas sobre a obra.