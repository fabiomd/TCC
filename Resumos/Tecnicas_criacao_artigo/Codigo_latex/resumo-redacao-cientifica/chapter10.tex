\chapter{Como Elaborar Referências Bibliográficas (NBR 6023:2002)}

\section{Conceito}

\paragraph{}
Refência bibliográfica é o conjunto de elementos que permitem a identificação de documentos impressos em variados tipos de material.

\section{Elementos essenciais e complementares}

\paragraph{}
São elementos essenciais de uma referência

\subparagraph{}
autor(es)
\subparagraph{}
título e subtítulo
\subparagraph{}
edição
\subparagraph{}
local
\subparagraph{}
editora
\subparagraph{}
data de publicação

\paragraph{}
Elementos complementares nas referências a livros são:

\subparagraph{}
ilustrador
\subparagraph{}
tradutor
\subparagraph{}
revisor
\subparagraph{}
adaptador
\subparagraph{}
compilador
\subparagraph{}
número de páginas
\subparagraph{}
volume
\subparagraph{}
ilustrações
\subparagraph{}
dimensões
\subparagraph{}
série editorial ou coleção
\subparagraph{}
notas
\subparagraph{}
ISBN
\subparagraph{}
índice

\section{Citação de monografias (livros, separatas, dissertações)}

\paragraph{}
São elementos imprescindíveis: autor, título da obra, edição, local, editor e ano de publicação.